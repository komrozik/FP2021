\section{Diskussion}
In diesem Versuch konnte mithilfe von primitiven Resonatoren einem Lautsprecher und
einem Mikrofon eindrucksvoll das Verhalten quantenmechanischer Systeme betrachtet werden.
Somit konnten mit einem Kugelresonator Schlüsse auf Atomorbitale, Aufspaltung der Resonanzen bei Symmetriebrechungen und deren Verhalten eines
Wasserstoffatoms analysiert werden und somit die Quantenzahlen $l$ und $m$ bestimmt werden.\\
Durch die Kopplung zweier Kugelresonatoren durch eine Blende konnten Rückschlüsse auf das Wasserstoffmolekül gezogen werden.
So zeigt sich, dass sich die Resonanzfrequenz der ersten und zweiter Ordnung nicht durch eine sich ändernde Kopplungsstärke (Blendendurchmesser)
beeinflussen lässt und somit der Entartung in $m$ entspricht. Zudem zeigt sich die Proportionalität
der Resonanzfrequenz der dritten Ordnung zum Blendendurchmesser und ist somit nicht entartet.
Durch die Bestimmung der relativen Phasenverschiebung der Resonatorfrequenzen durch das Betrachten der winkelabhängigen Druckamplitude, konnten zwei der drei Resonanzen
einem Molekülzustand zugeordnet werden.\\
Bei Verwendung von Rohrzylindern konnte auf grundlegende Eigenschaften von 1-dim-Festkörpern zurückgeschlossen werden. So würden Bändern, Bandlücken,
Verhalten der Elektronen bei sich ändernden Potenzialen sichtbar. Über das Einführen eines Störzylinders
in die Resonatorkette konnte eine Fehlstellung im Gitter des Festkörpers simuliert werden. Darüber hinaus konnte
eine zwei-atomige Basis eines Festkörpers analysiert werden, indem abwechselnde Blenden und somit unterschiedlich starke Kopplungen im Festkörper,
in die Resonatorkette eingebaut wurden und schlussendlich der Einfluss einer übergeordneten Periodizität
des Gitters und den daraus resultierenden neunen Bandlücken beobachtet werden.
\label{sec:Diskussion}