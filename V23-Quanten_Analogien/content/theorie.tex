\section{Ziel}
Zeil ist die Durchführung von akustischen Experimenten mit einem Kugelresonator und verschiedenen Resonatorketten
und der Vergleich mit quatnenmechanischen Systemen eines Wasserstoffatoms, Wasserstoffmoleküls und 1-dimensonalen-Festkörpern.

\section{Versuchseinführung}
Im Folgenden Versuch wird mit hilfe eines Mikrofon und eines Lautsprechers die Druckverteilung in einem Rohr- und Kugelresonator 
betrachtet. Dazu wird ein jeweils das Frequenzsprektrum für die jeweiligen Resonatoren betrachtet, die durch ein Frequenz zu Spannung Konverter
mittels einer geeigneten Software erstellt werden. 

\section{Theorie}
Bei den vermessenden Reonsatoren kann ein Vergleich zu quantenmechanischen Modellen gezogen werden. In diesem 
Abschnitt sollen die Gemeinsamkeiten zwischen den im Versuch betrachteten Modellen und ihren
quantenmechanischen Analogien aufgezeigt werden.

Die Druchverteilung in einem Resonator kann mithilfe der Helmholtzgleichung
\begin{equation}
    \Delta \varphi=\lambda \cdot \varphi
    \label{eq:helmholtz_all}
\end{equation}
beschrieben werden. Diese Differentzialgelichung beschreibt die Druckänderung mit den Randbedingungen
\begin{equation}
    \Delta=\sum_{k=1}^n \frac{\partial}{\partial x_k^2}
\end{equation}
des Resonatorrand $\delta \Omega$.




\subsection{Der Rohrresonator}
Der Rohrresonator wird an einem Ende mit dem Mikrofon und am anderen Ende mit dem Lautsprecher abgeschlossen.
Beim abspielen einzelner Frequenzen entsteht genau dann eine Resonanz, wenn die Wellenlänge $\lambda$ der Schallwelle
mit der Reflexion der Welle konstruktiv Inferferieren. Dies geschied genau dann, wenn gilt
\begin{equation}
    L=n\frac{\lambda}{2}n\frac{c}{2f}.
\end{equation}
Wobei $L$ die Rohrlänge, $c$ die Schallgeschwindikeit und $f$ die Frequenz der Schallwelle und $n$ ein ganzahliges Vielfaches ist.
Im Rohr findet dabei nur eine waagerechte Ausbreitung der Luftmoleküle zur Ausbreitungsrichtung der Schallwelle statt.
Die Druckverteilung $P(x,t)$ ergibt sich dann mit der Gleichung \ref{eq:helmholtz_all} zu
\begin{equation}
    \frac{\partial^2}{\partial t^2}P(x,t)=\frac{1}{\rho_\kappa}\frac{\partial^2}{\partial x^2}P(x,t)
\end{equation}
\label{sec:Theorie}
wobei $\rho$ die Dichte und $\kappa$ die KOmpressiblität des Mediums is indem sich die Schallwellen ausbreiten (hier: Luft).
Daraus folgt die zeitabhängige Lösung
\begin{equation}
    P(x,t)=p(x)\cdot \cos{(wt)}
    \label{eq:rohr_loe}
\end{equation}
mit der Ortsanteil $p(x)$.
Die Kreiszahl $k$ lässt sich über die Beziehung $k=2\pi/\lambda$ zu
\begin{equation}
    k=\frac{n\pi}{L}
\end{equation}
bestimmen.

Der Vergleich mit einem Teilchen im unendlichen Potenzialtopf ist dabei ein Analogon, 
den Materie (bzw. Elektronen und Protonen) besitzen ebenfalls Wellencharakter die durch die de-Broglie-Wellenlänge $\lambda_B$ über
\begin{equation}
    \lambda_B=\frac{h}{p}
\end{equation}
mit dem plankschen Wirkungsquantum $h$ und dem Impuls des Teilchens $p$ beschreiben werden kann.
Die Differentzialgeleichung für die Wellenfunktion $Rho(x,t)$ für das Teilchen im Kasten folgt dabei aus der Schrödingergleichung
\begin{equation}
    i\hbar \frac{\partial}{\partial t}\Rho(x,t)=-\frac{\hbar^2}{2m}\frac{\partial^2}{\partial x^2}\Rho(x,t)+V(x)\Rho(x,t)
\end{equation}
mit der teilchenmasse $m$, dem Potenzial $V(x)$. Für den unendlichen Potenzialtopf gilt $V(0 < x < L)=0$ und
$V(x \geq L)\rightarrow \infty$ und $V(x \leq 0)\rightarrow \infty$.
Damit ergibt sich die Lösung
\begin{equation}
    \Rho(x,t)=\rho(x)\cdot \exp{(-i\omega t)}
\end{equation}
und die zeitunabhängige Schrödingergleichung
\begin{equation}
    E\rho(x)=\frac{\hbar^2}{2m}\frac{\partial^2}{\partial x^2}\rho(x)
\end{equation}
mit der Energie $E$.
Für eine stehende Welle folgt daraus die Gleichung der Form
\begin{equation}
    \rho(x)=A\sin{(kx+\phi)}
    \label{eq:quant_loe}
\end{equation}
mit der komplexen Amplitude $A$, der Kreiszahl $k$ und der Phase $\phi$.
An den Rändern ergibt sich dabei eine Wellenfunktion von null, sodass aus dieser Randbedingung
$\rho(x=0)=0$ und $\rho(x=L)=0$ folgt
\begin{equation}
    k=\frac{n\pi}{L}.
    \label{eq:k}
\end{equation} 

Es wird deutlich das der Vergleich der beiden Modelle berrechtigt ist. Im klassischen Fall erzeugt man eine kosinusabhängige stationäre
Druckverteiung (vgl. Gl. \ref{eq:rohr_loe}), während das quantenmechanische Modell 
eine stationäre Wahrscheinlichkeitsverteilung ergibt (vgl. Gl. \ref{quant_loe}). Unterschiede zeigen sich bei den Randbedingungen.
Denn während bei dem quantenmechanischen Teilchen die Wahrscheinlichkeitsdichte verschwindet, sind an den Enden des Rohres 
die Druchunterschiede maximal, was allerdings kein Einfluss auf die erlaubten Wellenlängen hat.
Beide Modelle bilden dabei stehende Wellen für die Kreiszahl $k$ nach Gleichung \ref{eq:k} aus.

\subsection{Der Kugelresonator}
Nun wird ein Resonator mit einer Kugelform betrachtet (Hohlkugel), der aus zwei Halbkugel besteht, die gegen einander verdreht werden können.
Somit können die Resonanzamplituden in Abhängigkeit des Winkels $\alpha$ gemesen werden.

Zur genauen Beschreibung der dreidimensionalen Problems der Druckverteilung wird nun die Kugelsymmetrie mit den Koordinaten des Polarwinkels $\theta$ (0 bis $\pi$)
und des Azimutwinkel $\varphi$ (0 bis $2\pi$)verwendet.
Analog zum Rohrrensonator folgt aus der Helmholtzgleichung
\begin{equation}
    \frac{\partial^2 P(\vec{r}{,}t)}{\partial t^2}=\frac{1}{\rho\kappa}\Delta P(\vec{r}{,}t)
\end{equation}
Mit dem Ansatz $P(\vec{r}{,}t)=p(\vec{r})\cos{(\omega t)}$ ergibt sich die stationäre 
Druckverteilung nach
\begin{equation}
    -\frac{w^2}{c^2}p(\vec{r})=\Delta p(\vec{r})
\end{equation}
die mit dem Laplace-Operator $\Delta$ in Kugelkoordinaten zu einem Winkel- $Y_l^m(\theta{,}\varphi)$ und 
Radialanteil $f(r)$ seperiert werden kann. Dabei ist $Y_l^m(\theta{,}\varphi)$ die Kugelflächenfunktion, welche die 
stationäre, zeitunabhängige Druckverteilung beschreibt. Dabei ist $l$ die Drehimpulsquantenzahl ($0\leq l \leq n-1$),



