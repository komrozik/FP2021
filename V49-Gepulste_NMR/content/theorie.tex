\section{Ziel}
In diesem Versuch soll durch die gepulste Kernspinresonanz die magnetischen Resonanzen an Protonen ermittelt werden.
Dazu zählt die messung der Spin-Gitter-Relaxationszeit, der Spin-Spin-Relaxationszeit und der Difusionskoeffizient.

\section{Theorie}
\subsection{Kernspin}
Protonen und Neutronen gehören zu der Gruppe der Fermionen, die über einen halbzahligen Spin $\vec{I}$ definiert sind.
Im Verbund innerhalb eines Atoms resultieret daraus der Gesamtkernspin, der sich mittels des magnetischen
Dipolmoments schreiben lässt als
\begin{equation}
    \vec{\mu_I}=\vec{I}\cdot\gamma
\end{equation}
Dabei ist $\gamma$ das gyromagnetische Verhältnis und $\vec{I}$ der Gesamtdrehimpuls (oder auch Kernspin genannt) mit
\begin{equation}
    |\vec{I}|=\sqrt{I(I+1)}\hbar
\end{equation}
Befindet sich der Kern in einem exterenen MAgnetfeld $\vec{B_0}$ so kommt es aufgrund der Drehimpulserhaltung zu einer
Wechselwirkung zwischen dem magnetischen Moment $\vec{\mu_I}$ und dem Feld.
Daraus resultiert eine Aufspaltung der Energieniveaus.\\
Ist das Magnetfeld bespielsweise in z-Richtung ausgebreitet bleiben die Kernspin-Komponenten
$\vec{I_x}$ und $\vec{I_y}$ bei zeitlicher Mittlung gleich $0$. Lediglich die $I_z$-Komponente folgt
\begin{align}
    I_z&=\hbar m_I &\text{mit } m_I&=-I,...,I
\end{align} 
Für die Energieaufspaltung die zur ($2I+1$) diskreten Energieniveaus führt folgt dabei
\begin{equation}
    \Delta E=\gamma \hbar B_0=E_{m_I}-E_{m_{I+1}}
\end{equation}
\subsubsection*{Beispiel Wasserstoffkern}
Besonders anschaulich ist dies für den Wasserstoffkern, da dieser lediglich aus einem einzelnen Proton besteht.
Für den Kernspin gilt $I=\pm\frac{1}{2}$ aus dem zwei Energieniveaus resultieren.
Je nach Ausrichtung der Spins gegenüber dem angelegten Magnetfeld ist der Zustand energiereicher (antiparallele Ausrichtung)
oder energieärmer (parallele Ausrichtung). Für andere Ausrichtungen folgt aufgrund der Drehimpulserhaltung, ähnlich wie bei einem Kreisel, die
Präzession mit der (Larmor-)Frequenz 
\begin{equation}
    \omega_L=\gamma B_0
    \label{eq:lamor}
\end{equation}

Da sich die einzelne Atome meist schwer isolieren lassen, werden im allgemeinen Atom-Anhäufungen (Ensembles)
betrachtet. Die Besetzung der Energieniveaus erfolgt demnach nach der Boltzmannverteilung gemäß
\begin{equation}
    N_m \propto \exp\left(\frac{E_m}{k_BT}\right)
\end{equation}
Nun ergibt sich daraus eine messbare makroskopische Magnetisierung $M_0$ der Probe, welche die Grundlage der NMR-Spektroskopie bildet.
\begin{equation}
    \vec{M_0}=\frac{N\gamma I(I+1)}{3k_BT}\vec{B_0}
\end{equation}
wobei $N$ die Anzahl der Teilchen pro Volumeneinheit und $T$ die absolute Temperatur darstellen.
\subsection*{Kernspinresonanz}
Legt man nun ein zusätzliches magnetisches Wechselfeld senkrecht zu $B_0$ an, so lässt die die Richtung der Magnetsierung $\vec{M_0}$
manipulieren. Für eine genauere Beschreibung verwende man hierbei ein rotierendes Koordinatensystem (KOS) welches mit der 
Lamor-Frequenz $\omega_L$ aus Gleichung \ref{eq:lamor} um $\vec{B_0}$ rotiert. $M_0$ erscheint nun fortan in fester Richtung.
Das neue in x-Richtung ausgesendete magnetische Wechselfeld bildet im KOS ein konstanten Vektor $\vec{B_1}$.
Die Manipulation durch $B_1$ führt zu einer Präzession in y-z-Ebene. Schaltet man das Feld wieder aus, so beobachtet man ein Drehung
der Magnetsierung um den Winkel $\theta$.\\
Für die Anwendung sind die Einwirkzeiten zu einem 90⁰- und 180°-Puls besonders interessant.
\subsection{Prozesse}
Durch das Ausschalten das Feldes $B_1$ wird der Gesamtdrehimpuls nun nicht mehr beeinflusst und das ein niederiger Energiezustand
(Parallelzustand) wird angestrebt. Das System "präzediert" in den Grundzustand zurück. Man spricht hierbei von Relaxation.
\subsubsection{Spin-Spin-Relaxation}
Für einen 90°-Puls (transversalen Puls) kippt nun die makroskopische Magnetisierung aus dem them. Gleichgewicht in die x-y-Ebene.
Durch eine klassische Beschreibung mittels der Bloch-Formeln, lässt sich die Zeit bzw. die Zerfallskonstante $T_2$ bestimmen,
die es benötigt um die entstandende Quermagnetisierung durch die Manipulation durch $B_1$ abzubauen.
Dabei ändert sich gemäß der Drehimpulserhaltung die Gesamtenergie des Spinsystems (oder des Gitters) nicht. Lediglich die Kohärenz geht verloren, weshalb
es sich hierbei um einen teilsweise irreversiblen Prozess handelt und somit von Entropieprozess gesprochen wird.\\

Die Bloch-Formeln im KOS nehmen die Form
\begin{align}
    \frac{dM_x}{dt}&=-\frac{M_x}{T_2}\\
    \frac{dM_y}{dt}&=-\frac{M_y}{T_2}
\end{align}
an. Durch die Integration über die Zeit lässt sich die Zerfallskonstante/transversale Relaxationszeit $T_2$ aus
\begin{equation}
    M(t)=M_0\exp\left(-\frac{t}{T_2}\right)
\end{equation}
bestimmen.
\subsubsection{Spin-Gitter-Relaxation}
Für einen 180°-Puls befindet sich das System im energiereichsten Zustand. Wie bereits erwähnt sind hierbei nun die antiparallelen Zustand
häufiger besetzt als die parallelen Zustand. Auch dieses Spinsystem 'kühlt' nun fortlaufend ab, bis die angeregten Zustände gemäß der Boltzmannverteilung
in die Grundzustands-Verteilung zurück kommen. Dabei geben die Kerne ihre Energie an das Gitter ab. 
Diese Zeitspanne $T_1$ der longitudinalen Relaxation folgt aus der Bloch-Formel
\begin{equation}
    \frac{dM_z}{dt}=-\frac{M_z-M_0}{T_1}
\end{equation} 
und folgt über die zeitliche Interation aus
\begin{equation}
    M_z(t)=M_0\left(1-2\exp\left(\frac{-t}{T_1}\right)\right).
\end{equation}