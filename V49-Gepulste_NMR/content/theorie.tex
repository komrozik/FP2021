\section{Ziel}
In diesem Versuch soll durch die gepulste Kernspinresonanz die magnetischen Resonanzen an Protonen ermittelt werden.
Dazu zählt die messung der Spin-Gitter-Relaxationszeit, der Spin-Spin-Relaxationszeit und der Difusionskoeffizient.

\section{Theorie}
Protonen und Neutronen gehören zu der Gruppe der Fermionen, die über einen halbzahligen Spin $\vec{I}$ definiert sind.
Im Verbund innerhalb eines Atoms resultieret daraus der Gesamtkernspin, der sich mittels des magnetischen
Dipolmoments schreiben lässt als
\begin{equation}
    \vec{\mu_I}=\vec{I}\cdot\gamma
\end{equation}
Dabei ist $\gamma$ das gyromagnetische Verhältnis und $\vec{I}$ der Gesamtdrehimpuls (oder auch Kernspin genannt) mit
\begin{equation}
    |\vec{I}|=\sqrt{I(I+1)}\hbar
\end{equation}
Befindet sich der Kern in einem exterenen MAgnetfeld $\vec{B_0}$ so kommt es aufgrund der Drehimpulserhaltung zu einer
Wechselwirkung zwischen dem magnetischen Moment $\vec{\mu_I}$ und dem Feld.
Daraus resultiert eine Aufspaltung der Energieniveaus.\\
Ist das Magnetfeld bespielsweise in z-Richtung ausgebreitet bleiben die Kernspin-Komponenten
$\vec{I_x}$ und $\vec{I_y}$ bei zeitlicher Mittlung gleich $0$. Lediglich die $I_z$-Komponente folgt
\begin{align}
    I_z&=\hbar m_I &\text{mit } m_I&=-I,...,I
\end{align} 
Für die Energieaufspaltung die zur ($2I+1$) diskreten Energieniveaus führt folgt dabei
\begin{equation}
    \Delta E=\gamma \hbar B_0=E_{m_I}-E_{m_{I+1}}
\end{equation}
\subsubsection*{Beispiel Wasserstoffkern}
Besonders anschaulich ist dies für den Wasserstoffkern, da dieser lediglich aus einem einzelnen Proton besteht.
Für den Kernspin gilt $I=\pm\frac{1}{2}$ aus dem zwei Energieniveaus resultieren.
Je nach Ausrichtung der Spins gegenüber dem angelegten Magnetfeld ist der Zustand energiereicher (antiparallele Ausrichtung)
oder energieärmer (parallele Ausrichtung). Für andere Ausrichtungen folgt aufgrund der Drehimpulserhaltung, ähnlich wie bei einem Kreisel, die
Präzession mit der (Larmor-)Frequenz 
\begin{equation}
    \omega_L=\gamma B_0
\end{equation}

Da sich die einzelne Atome meist schwer isolieren lassen, werden im allgemeinen Atom-Anhäufungen (Ensembles)
betrachtet. Die Besetzung der Energieniveaus erfolgt demnach nach der Boltzmannverteilung gemäß
\begin{equation}
    N_m \propto \exp\left(\frac{E_m}{k_BT}\right)
\end{equation}
Nun ergibt sich daraus eine messbare makroskopische Magnetisierung $M_0$ der Probe, welche die Grundlage der NMR-Spektroskopie bildet.
\begin{equation}
    \vec{M_0}=\frac{N\gamma I(I+1)}{3k_BT}\vec{B_0}
\end{equation}
wobei $N$ die Anzahl der Teilchen pro Volumeneinheit und $T$ die absolute Temperatur darstellen.
\subsection*{Kernspinresonanz}
