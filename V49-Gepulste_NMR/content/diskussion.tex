\section{Diskussion}
\label{sec:Diskussion}
Für die jeweiligen Relaxationszeiten ergeben sich folgende Messergebnisse und damit Abweichungen von den Literaturwerten:
\begin{align*}
    T_1 & = \SI{2.784 +- 0.04}{\second} & T_2 & = \SI{0.95 +- 0.08}{\second}\\
    T_{1,\text{Lit}} & = \SI{3.09}{\second} & T_{2,\text{Lit}} & = \SI{1.52}{\second}\\
    \Delta T_1 & = \SI{9.9+-1.3}{\percent} & \Delta T_2 & = \SI{37+-5}{\percent}\\
\end{align*}
Mit Abweichungen unter \SI{40}{\percent} können beide Messungen als Erfolgreich angesehen werden.
Besonders die Messung der Spin-Gitter-Relaxationszeit ist mit einem Fehler von \SI{9.9}{\percent} als erfolgreich anzusehen.
Der kleine Fehler kann aus Ableseunsicherheiten oder systematischen Fehlern des Aufbaus stammen.
Der deutlich größere Fehler der Spin-Spin-Relaxationszeit kann auf falsche Einstellungen der Technik und auf unsicherheiten beim Ablesen der Werte zurückgeführt werden.

Die Messung der Diffusionskonstante mit einem Wert von .\SI{59.58 +- 1.29 E-09}{\metre \squared  \per \second} weicht stark vom Literaturwert \SI{2.36 E-09}{\metre\squared\per\second} ab.
Diese große Abweichung ist wahrscheinlich auf eine schlechte Bestimmung des Feldgradienten kombiniert mit dem Fehler der Diffusionszeit zurück zu führen.

Der Molekülradius wird mit diesem fehlerbehafteten Wert zu \SI{358 +- 7 E-14}{\metre} bestimmt und weicht damit \SI{97.94 +-0.04}{\percent} vom Literaturwert \SI{1.74 E-10}{\metre} ab.