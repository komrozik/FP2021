\section{Diskussion}
\label{sec:Diskussion}

Der Graph in Abbildung \ref{fig:rate_20Hz} sollte statt eines Peaks ein Plateau aufweisen.
In unserer Messung ist dieses Plateau nicht gut zu erkennen.
Ein Plateau könnte bei den Werten rund um die eine Verzögerung von Null liegen, allerdings spricht der maximale Wert bei einer Verzögerung von \num{0.5} nicht dafür, da er selbst mit beachtung der Unsicherheit nur schwierig auf ein Plateau mit den anderen Werten zu platzieren ist.
Durch die geringe Anzahl an Messwerten an den Randwerten der Messung, ist es möglich, dass ein Plateau existiert und die Ränder dieses Plateaus nicht erkennbar sind.

Die Wahl der Verzögerung von \num{0.5} wurde mithilfe des Maximalwertes getroffen und könnte zu einem potentiellen Plateau passen, lässt sich allerdings nicht mit der Gauß-Funktion validieren.
Die Gauß-Funktion hat einen Maximalwert bei einer Verzögerung von Eins.

Mithilfe der Näherung der Exponential-Funktion wurde die Lebensdauer der Myonen zu \SI{1.70522+-0.05722}{\micro\second} bestimmt, mit einer relativen Abweichung zum Theoriewert von \num{0.22384 +- 0.02605} also circa \SI{22.38}{\percent}.
Die starke Abweichung der Werte kann auf die Falsche Wahl der Verzögerung und auf eine schlechte Datenaufnahme bei der Kalibration des Aufbaus zurückgeführt werden.