\newpage
\section{Anhang}
\label{sec:Anhang}
\begin{figure}[H]
    \centering
    \includegraphics[width = 0.75\textwidth]{./plots/rate_20Hz_plateau.pdf}
    \caption{Versuch eine Plateaufunktion an die Daten zu Fitten. Die Plateaufunktion ist in Formel \ref{eqn:plateau} gezeigt. Die Plateaufunktion ist in diesem Fall eine Potenz der Gaußfunktion.}
    \label{fig:plateau}
\end{figure}
\begin{equation}
    a \cdot e^{-\left(\frac{x-b}{2c}\right)^{64}}
    \label{eqn:plateau}
\end{equation}

\begin{figure}[H]
    \centering
    \includegraphics[width = 0.75\textwidth]{./plots/rate_20Hz_plateau2.pdf}
    \caption{Versuch eine Plateaufunktion an die Daten zu Fitten. Die Plateaufunktion ist in Formel \ref{eqn:plateau2} gezeigt. Die Plateaufunktion besteht aus einer Kombination von zwei Exponentialfunktionen.}
    \label{fig:plateau2}
\end{figure}
\begin{equation}
    \frac{c}{ \left(e^{b\left(x-a\right)} +1\right) \left(e^{b\left(-x-a\right)} +1\right) }
    \label{eqn:plateau2}
\end{equation}
Mit der Plateaufunktion aus Formel \ref{eqn:plateau2} wird ein Plateau der Breite $2\cdot a$ erzeugt.
Wie Steil das Plateau ist wird durch den Parameter $b$ verändert und die Höhe des Plateaus ist $c$.