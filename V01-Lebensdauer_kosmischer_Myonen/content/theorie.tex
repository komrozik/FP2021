\section{Ziel}
In diesem Versuch soll die Lebensdauer kosmischer Myonen vermessen werden.
\section{Theorie}
\label{sec:Theorie}
\subsection{Kosmische Myonen}
Myonen sind Elemntarteilchen mit einer Masse von $0,106 \frac{GeV}{c^2}$ und einer Ladung von $-1 e$.
Die Myonen entstehen aus Wechselwirkungen von kosmischen Teilchen in der Atmosphäre.
Bei der kosmischen Strahlung unterscheidet man zwischen zwei Arten, der primären kosmischen Strahlung und der sekundären kosmischen Strahlumg.
Die primäre kosmische Strahlung besteht hauptsächlich aus Protonen (ca 85\%) und zu 14\% aus Heliumkernen. Die restlichen 