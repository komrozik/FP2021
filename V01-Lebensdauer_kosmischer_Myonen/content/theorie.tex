\section{Ziel}
In diesem Versuch soll die Lebensdauer kosmischer Myonen vermessen werden.
\section{Theorie}
\label{sec:Theorie}
\subsection{Kosmische Myonen}
Myonen sind Elemntarteilchen mit einer Masse von $0,106 \frac{GeV}{c^2}$ und einer Ladung von $-1 e$.
Die Myonen entstehen aus Wechselwirkungen von kosmischen Teilchen in der Atmosphäre.
Bei der kosmischen Strahlung unterscheidet man zwischen zwei Arten, der primären kosmischen Strahlung und der sekundären kosmischen Strahlumg.
Die primäre kosmische Strahlung besteht hauptsächlich aus Protonen (ca 85\%) und zu 14\% aus Heliumkernen.
Die Entstehungskanäle der kosmischen Myonen:
\begin{equation*}
    \feynmandiagram [horizontal=a to b] {
        i1 [particle=\(u\)] -- [fermion] a -- [fermion] i2 [particle=\(\overline d\)],  
        a -- [boson, edge label'=\(W^{+}\)] b,
        f1 [particle=\(\mu^{+}\)] -- [fermion] b -- [fermion] f2 [particle=\(\nu_{\mu}\)], 
    };
    \hspace{4cm}
    \feynmandiagram [horizontal=a to b] {
        i1 [particle=\(d\)] -- [fermion] a -- [fermion] i2 [particle=\(\overline u\)],  
        a -- [boson, edge label'=\(W^{-}\)] b,
        f1 [particle=\(\overline \nu_{\mu}\)] -- [fermion] b -- [fermion] f2 [particle=\(\mu^{-}\)], 
    };
\end{equation*}
Diese Zerfälle sicnd auch mit Kaonen ähnlich möglich aber deutlich seltener.


\begin{equation*}
    \feynmandiagram [layered layout, horizontal=a to b] {
        a [particle=\(\mu^{-}\)] -- [fermion] b -- [fermion] f1 [particle=\(\nu_{\mu}\)],
        b -- [boson, edge label'=\(W^{-}\)]c,
        c -- [anti fermion] f2 [particle=\(\overline \nu_{e}\)],
        c -- [fermion] f3 [particle=\(e^{-}\)],
    };
    \hspace{4cm}
    \feynmandiagram [layered layout, horizontal=a to b] {
        a [particle=\(\mu^{+}\)] -- [anti fermion] b -- [anti fermion] f1 [particle=\(\overline \nu_{\mu}\)],
        b -- [boson, edge label'=\(W^{+}\)] c,
        c -- [anti fermion] f2 [particle=\(e^{+}\)],
        c -- [fermion] f3 [particle=\(\nu_{e}\)],
    };
\end{equation*}
\subsection{Vorbereitung}
\subsubsection{Was definiert die Lebensdauer und aus welchem Zusammenhang lässt sie sich herleiten?}
\begin{align*}
    \frac{1}{e} N_0 = N_0 e^{-\frac{\Gamma}{\hbar} t}
\end{align*}
\subsubsection{Berechnen sie klassich und relativistisch aus der Sicht eines auf der Erde ruhenden Beobachters die Reichweite eines kosmischen Myons.}
\begin{align*}
    E_{\mu} &= \SI{10}{\giga \electronvolt}\\
    E^2 &= m^2c^2 + p^2c^2 \\
    v &= c\sqrt{1-\frac{m^2 c^4}{E^2}}\\
    s &= v\tau = \tau c \sqrt{1-\frac{m^2 c^4}{E^2}} = \SI{659.5}{\metre}\\
    s\prime &= \frac{s}{\gamma} = \SI{659.5}{\metre} \sqrt{1-\frac{v^2}{c^2}} = \SI{62418}{\metre}
\end{align*}
\subsubsection{Welche Ereignisrate erwarten Sie auf der Erdoberfläche? Den Szintillatortank können Sie als liegenden Zylinder nähern.}
Die Myonen werden als senkrecht auf den Szintillator treffend angenommen und die Fläche des Tanks ergibt sich damit zu $A = 2hr$.
Auf der Erdoberfläche beträgt der Myonenfluss circa \SI{1}{\per \square \centi \metre \per \minute}.
\begin{align*}
    \SI{1}{\per \square \centi \metre \per \minute} &= \SI{166.7}{\per \square \metre \per \second} \\
    \pi h r^2 &= 2\pi r^3 = \SI{0.05}{\cubic \metre} \\
    r &≈ \SI{0.1996}{\meter} \\
    h &= 2r \\
    \text{Ereignisrate} &= \SI{166.7}{\per \square \metre \per \second} \cdot 4r^2 \\
    &= \SI{1595}{\per \minute} \\
    &= \SI{26.58}{\hertz}
\end{align*}