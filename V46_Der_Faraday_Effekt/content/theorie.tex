\section{Ziel}
In deisem Experiment soll der Faraday Effekt in GaAs gemessen werden.
\section{Theorie}
\label{sec:Theorie}
\subsection{Bandstruktur und effektive Masse}
Die Verteilung von Elektronen in Festkörpern wird üblicherweise durch die sogenannte Bandstruktur dargestellt und soll hier noch einmal kurz wiederholt werden.
Grundsätzlich unterscheidet man bei Festkörpern zwischen Metallen, Halbleitern und Isolatoren.
Jede Art von Festkörper besitzt ein Leitungsband in dem sich die Leitungselektronen befinden und Valenzbänder für die Valenzelektronen.
Die verschiedenen Festkörper unterscheiden sich in der Bandstruktur durch die verschieden großen Abstände zwischen Leitungs und Valenbändern.
Bei metallen liegt die Fermi Energie in dem Leitungsband und die Bandlücke zwischen Leitungs- und Valenzband ist relativ klein, daher sind Metalle gute elektrische Leiter.
Bei Halbleitern und bei Isolatoren liegt die Fermienergit in der Bandlücke zwischen Leitungs und Valenzband. 
Isolatoren haben die größte Bandlücke und sind damit auch die schlechtesten elektrischen Leiter. 
Bei Halbleitern ist die Bandlücke relativ klein und für die Elektronen gut überwindbar.

Um Elektronen im Festkörper zu beschreiben ist zu beachten, dass das Elektron nicht nur ein elektrisches Potential erfährt sondern zusätzlich noch von einem ortsabhängigem Potential beeinflusst wird.
Damit die Elektronen weiterhin als freie Elektronen beschrieben werden können modifiziert man ihre Masse zu einer effektiven Masse $m^\*$ die das Kristallpotential mit beachtet.

\subsection{Dotierung von Halbleitern}
Die Dotierung von Halbleitern meint das Einfügen von Fremdatomen in das bestehende Kristallgitter um die Bandstruktur und damit auch die Leitfähigkeit des Kristalls zu verändern.
Es wird bei der Dotierung zwischen Donatoren und Akzeptoren unterschieden.

\subsection{Faraday Effekt}
Der Faraday Effekt beschreibt die Rotation der Polarisationsebene eines Lischtstrahls beim Durchlaufen einer !!Probe!! in einem Magnetfeld.

\subsection{Zirkulare Doppelbrechung}
Polarisationsebene eines linear polarisierten Lichtstrahls wird beim durchlaufen eines Kristalls gedreht.

\subsection{Bestimmung der Effektiven Masse}